% !TEX encoding = UTF-8 Unicode
% !TEX TS-program = XeLaTex
% !TEX root = EMU2016_booklet.tex

{\fontsize{30}{30} \svolk{\emph{Per volontà e \newline per caso}}}

\begin{quote}
\begin{it}
	\svolk{Pierre Boulez, artista e teorico, ha segnato profondamente la vita culturale europea e la sua recente scomparsa mette ancor più in evidenza il ruolo propulsivo del suo lavoro.

La frase \textbf{Par volonté et par hasard} (\textbf{Per volontà e per caso} in italiano), tratta da una intervista rilasciata negli anni settanta da Boulez al musicologo belga Célestin Deliège, condensa  e integra  l’attitudine sperimentale e di ricerca che caratterizza la musica colta dal secolo scorso a oggi.

EMUFest 2016, nel ricordare Boulez, intende perseguire un progetto etico, oltre che didattico, destinato ai­­ valori della ricerca musicale, dell’innovazione, dell’espressione libera, cosciente e sempre attenta al rapporto dell’arte con il sociale.  
Il Festival considera basilari questi valori e si propone di divulgarli attraverso ogni forma espressiva che, attraverso la musica, interpreta e stimola il pensiero contemporaneo.   

EMUFest è caratterizzato quest’anno da tre tipi di attività: la \textbf{Sequenza}, gli \textbf{Eventi} e le \textbf{Fusioni}.

La \textbf{Sequenza},  frutto del lavoro di selezione e interpretazione delle nuove opere, presenta una serie di 11 Concerti e performance in cui la musica dei maestri del ‘900 si alterna alle più recenti produzioni internazionali di giovani compositori.

Gli \textbf{Eventi}, realizzati in collaborazione con altri Conservatori italiani, Festival ed Enti di produzione e ricerca,  presentano, nel periodo Estate – Autunno 2016, conferenze, seminari e concerti di musicisti e studiosi di fama internazionale.

Le \textbf{Fusioni}, prodotte con la collaborazione di docenti e interpreti italiani ed esteri, presentano, in forma didattica e performativa, l’approccio alla composizione e alla esecuzione musicale contemporanea.}

\hfill \emph{Michelangelo Lupone}

\end{it}
\end{quote}