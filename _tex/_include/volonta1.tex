% !TEX encoding = UTF-8 Unicode
% !TEX TS-program = XeLaTex
% !TEX root = ../EMU2016_booklet.tex

\acusmatici{Ursula Meyer-K\"onig}
{Allears}{2012-13}{8'}

\medskip
    
\acusmatici{Benjamin O'Brien}
{Along the eaves}{2012-13}{8'20''}

\medskip
    
\acusmatici{Dennis Deovides A. Reyes III}
{Bolgia}{2014}{7'31''}

\medskip
    
\acusmatici{Dimitrios Savva}
{Balloon Theories}{2012-13}{14'30''}

\medskip
    
\acusmatici{Jones Margarucci}
{Inhabitated Places\textunderscore Part II}{2012-13}{5'52"}

\vfill

\descrizione{Allears}{Originariamente l'ispirazione per questo lavoro  proveniva da una serie di intense discussioni con persone non udenti o che hanno problemi d'udito. Abbiamo parlato dei pro e dei contro di apparati tecnici, come apparecchi acustici o impianti cocleari, le diverse risposte etiche ed emotive che le persone sentono, e i problemi d'identità che sollevano. 
Indossare apparecchi acustici cambia anche come i suoni vengono percepiti, a volte causando interferenze, distorsioni,percezione spaziale ridotta e troppo pieno di rumore.}

\descrizione{Along the Eaves}{ prende il nome dalla riga che di Franz Kafka "Incrocio". "Sulle notti di luna la sua passeggiata preferita è lungo la grondaia" Per comporre l'opera, ho sviluppato software personalizzato e utilizzato questi programmi in modi diversi per elaborare e sequenziare i miei materiali di base, che, in questo caso, include registrazioni audio d’acqua, bambini, e di strumenti a corda. Il mio interesse è quello di creare coincidenze sonore che suggeriscono i rapporti tra i suoni e le illusioni che promuovono.}

\descrizione{Bolgia}{è una parola italiana che significa "fossa" e "luogo chiassoso in cui regna la confusione". Questo termine è stato usato da Dante Alighieri nel suo noto lavoro letterario "Inferno". Bolgia è un brano stereofonico fisso per composizioni elettroacustiche, che illustra il viaggio di Alighieri nell'ottavo girone dell'inferno, e la sua esperienza in questo posto terribile. I gesti musicali e l'evento sonoro del pezzo evocano i diversi suoni e le diverse emozioni dell'inferno.}

\descrizione{Balloon Theories}{«Ho sempre trovato divertente strizzare palloncini, premerli con le dita fino allo scoppio... Non mi è mai interessato fino a quando non ho capito perché...»}

\descrizione{Inhabitated Places\textunderscore part II}{ è una composizione elettroacustica basata sul concetto di musica algoritmica. Sebbene la forma generale del brano sia stata determinata apriori e in modo convenzionale, tutti i suoni che ascoltiamo vengono scelti in tempo reale da vari algoritmi scritti in SuperCollider. Questi algoritmi selezionano in modo pseudocasuale dei samples da diverse cartelle e li riproducono a velocità diverse e in diversi momenti.  
È come se avessimo sistemato in una scatola (che in questo caso rappresenta la struttura dell’opera) degli oggetti in un dato ordine, ma ogni qual volta apriamo la scatola li troviamo disposti in modo differente da come li avevamo lasciati.}